\documentclass[a4paper,twocolumn]{article} % Document type

\ifx\pdfoutput\undefined
    %Use old Latex if PDFLatex does not work
   \usepackage[dvips]{graphicx}% To get graphics working
   \DeclareGraphicsExtensions{.eps} % Encapsulated PostScript
 \else
    %Use PDFLatex
   \usepackage[pdftex]{graphicx}% To get graphics working
   \DeclareGraphicsExtensions{.pdf,.jpg,.png,.mps} % Portable Document Format, Joint Photographic Experts Group, Portable Network Graphics, MetaPost
   \pdfcompresslevel=9
\fi

\usepackage{amsmath,amssymb}   % Contains mathematical symbols
\usepackage[ansinew]{inputenc} % Input encoding, identical to Windows 1252
\usepackage[english]{babel}    % Language
%\usepackage[round,authoryear]{natbib}  %Nice author (year) citations
\usepackage[square,numbers]{natbib}     %Nice numbered citations
%\bibliographystyle{unsrtnat}           %Unsorted bibliography
\bibliographystyle{plainnat}            %Sorted bibliography

\addtolength{\topmargin}{-30mm}% Removes 30mm from the top margin
\addtolength{\textheight}{30mm}% Adds it to the text height


\begin{document}               % Begins the document

\title{Reproducing Sequential Attention for Feature Selection}
\author{Gary Wang, Hank Lai \\ N16132087, N16131722 \\ n16132087@gs.ncku.edu.tw, n16131722@gs.ncku.edu.tw} 
%\date{2010-10-10}             % If you want to set the date yourself.

\maketitle                     % Generates the title




%%%%%%%%%%%%%%%%%%%%%%%%%%%%%%%%%%%%%%%%%%%%%%%%%%%%%%%%%%%%%%%%%%%%%%%%%%%%%%%%%%%
% Instructions regarding the report
%%%%%%%%%%%%%%%%%%%%%%%%%%%%%%%%%%%%%%%%%%%%%%%%%%%%%%%%%%%%%%%%%%%%%%%%%%%%%%%%%%%

\section{Introduction}
% Introduce motivation, background on feature selection, and overview of SA

\section{Problem Description}
% Describe the feature selection task and the problem SA solves

\section{Methods}
\subsection{Sequential Attention Algorithm}
% Description of algorithm, possibly Algorithm 1 from paper



\subsection{Experimental Setup}
% Dataset info, network architecture, optimizer, seeds, etc.

\section{Results}
\subsection{Dataset Overview}


\subsection{Feature Selection Accuracy}
% Accuracy table or plots for k=50 features


\subsection{Statistical Validation}
% t-test or p-value results table if needed

\section{Discussion}
% Analysis of differences, strengths, weaknesses, generalization, possible error sources

\section{Conclusion}
% Summary of what was reproduced and whether claims hold

\section*{References}


% Optionally include appendix
\clearpage
\appendix
\section*{Appendix}
\section{Hyperparameters}
% Learning rate, batch size, etc.

\section{Additional Figures}
% Supplementary plots

\end{document}      % End of the document
